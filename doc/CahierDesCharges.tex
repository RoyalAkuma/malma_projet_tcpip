\documentclass[a4paper 14pt]{article}
\usepackage[utf8]{inputenc}
\usepackage[T1]{fontenc}
\usepackage[francais]{babel}
\usepackage{graphicx}
\title{Cahier des charges, projet Reseau - M1 ALMA}
\author{Joachim CLAYTON, Arnaud GRALL}




\begin{document}

\maketitle
\date

\begin{figure}[h!]
   \includegraphics[width=350pt]{source/0.jpeg}
\end{figure}


\newpage
\renewcommand{\contentsname}{Sommaire} 
\tableofcontents
\newpage


\section{Introduction}



\subsection{Notre Groupe}

Nous sommes deux étudiants en premiere année du Master ALMA :
Arnaud Grall et Joachim Clayton.


\subsection{Problématique}

La problématique du projet est de réaliser une application qui utiliserait les échanges de données entre plusieurs terminaux.\\
Nous avons pour cela décider de créer un chat sur le model ''irc'' qui permetterai l'échange entre plusieurs clients grâce à un server distant.
De plus, cette application doit permettre de valider l'enseignement réalisé en module de Réseau et donc convenir à certaines normes et protocoles que nous choisirons.\\
Les choix de ces protocoles seront également détaillés et expliqués afin de valider la conception de notre logiciel.\\
Les spécifications de notre projet seront expliqués dans les parties suivantes.


\newpage


\section{Premiere Partie : Détails de l'application}

Cette partie décrit le rendu final attendu de notre application.

\subsection{Architecture de l'application}

L'application devra avoir pour architecture : 

\begin{enumerate}
	\item Un serveur central et distant ( c'est à dire que le serveur est détaché )
	\item Plusieurs ou une interface clients ( chaque client représente un hôte du réseau )
\end{enumerate}

\begin{figure}[h]
   \includegraphics[width=350pt]{source/1.png}
	\caption{Schéma intuitif architecture du chat.}
\end{figure}

\newpage

\subsection{UML}

Voici un schéma récapitulatif UML du logiciel :

%faut que tu mettes le schéma uml dans source et tu met le nom fichier à la place de 1.png ici :D
\begin{figure}[h]
   \includegraphics[width=350pt]{source/1.png}
	\caption{Schéma UML}
\end{figure}

\subsection{interface}

Notre projet dans un premier temps devra pouvoir fonctionner grâces à des terminaux.
Chaque terminaux lancé sera soit serveur (il n'y en aura qu'un) ou soit clients (il pourra y en avoir plusieurs différents).

\subsection{Choix des technologies}

Cette partie regroupe les différentes technologies que nous utiliserons.

\subsubsection{ncurses}

Premettant le rafraichissement des terminaux ??
% complette ça stp :D

\newpage
\subsubsection{protocole tcp/ip}

Afin de réaliser le transfert de donnée et des messages, nous avons décidé d'utiliser le protocol tcp/ip.

Ce protocole permet l'envoie des messages grâces aux adresses ip unique de chaque utilisateur. Il permet la viabilité des informations transmisent d'un terminal à un autre. Il permet aussi de vérifier si les données ont bien été transmise, entre autre :

TCP permet de :
\begin{enumerate}
	\item ,comme UDP, permet la gestion des ports.
	\item vérifier la capacité du destinataire à recevoir des données.
	\item segmenter les gros paquets de données en paquets plus petits pour que IP les accepte.
	\item numéroter les paquets, et à la réception de vérifier qu'ils sont tous bien arrivés.
	\item redemander les paquets manquants et de les réassembler avant de les donner aux logiciels.Des accusés de réception sont envoyés pour prévenir l'expéditeur que les données sont bien arrivées.
\end{enumerate}


\newpage
\section{Deuxieme Partie : Fonctionalités bonus}

Dans le cas ou notre application repondrait à nos attentes premieres, nous pensions travailler afin de développer ces fonctionalités bonus :

\subsection{Le transfert de Fichier}

Dans le cas du transfert de Fichier, nous pensions réaliser un transfert de fichier entre utilisateurs, toujours sur la base de tcp/ip, grâce ftp.

\subsubsection{FTP}

File Transfert Protocole est un protocole de communication destiné à l'echange de fichiers sur un réseau TCP/IP.
Il permet de copier un ou plusieurs fichier d'un terminal à un autre du réseau.
Il obéit à un model  Client-Serveur. 

\subsection{L'interface graphique}

\begin{figure}[h]
   \includegraphics[width=350pt]{source/2.png}
	\caption{Schéma de l'interface graphique}
\end{figure}

\newpage
\subsection{Message privé, message de groupe}

Nous pensions develloper des options qui permetterai l'envois de message à une personne en particulier ou à des groupes de personnes en particulier.

\subsection{Administrateur}

Cette fonctionalité permettrait l'emergence des clients administrateur qui se demarquerait du client.
Il aurait la possibilité d'avoir plus de commandes et de privilège qu'un client lambda.


\section{Conclusion}



\end{document}
